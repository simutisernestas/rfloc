The project is a step towards a general localization system for UAV. The experimental setup now have only included doing estimation on an arbitrary walking trajectory of a human for which dynamics cannot be identified. In case of a real system i.e. flying drone this wouldn't be the case, the dynamics are mostly known and mathematical model can be derived from first principles and used along side estimation problem to make more reasonable prediction steps. Besides that, drone usually is equipped with more sensors like IMU, barometer or camera which can provide additional information to make localization more robust. Probably even adding single IMU and using constant acceleration model instead of constant velocity model would improve the positioning accuracy tremendously.

Considering positioning performance only for this data modality and constant velocity model the results obtained are satisfactory and any improvements on filter side would be marginal. For instance, tunning of the EKF (process noise and measurements) could be done on collected data applying optimization techniques but for project scope is not relevant.

Available UWB chips have a number of operating modes with varying properties. The scope of possibilities was not explored in detailed here and could further be investigated for possible adjustment. There could be modes with more desirable properties to the application in consideration. This would include studying manufacturer provided data-sheets.

Work can could be extended in the future to be used on multiple agents acting in an environment together. For example, drone could act as an anchor and tag at the same time with single or two UWB devices mounted on top and provide distance measurements to neighboring agents around. This would require specialized algorithm to incorporate data coming in from moving "anchors" and more care to covariances of estimates because this would introduce cross dependencies between measurements (now anchor positions are considered to be static/deterministic and fully known).